% Declare any packages and options you are using here.

% Used packages. Keep packages in alphabetical order, except for hyperref below.
\usepackage{acro}  % Use acronyms (define actual acronyms in setup/acro.tex)
\usepackage{booktabs}  % Make tables look nicer
\usepackage{cite}  % Compress and sort multi-citations
\usepackage[inline]{enumitem}  % Better enumeration, including inline lists
\usepackage{graphicx}  % Allow figures
\usepackage{tabularx}  % Enable table columns to fill/wrap a specific width
\usepackage{xcolor}  % Define colors for comments, etc.
\usepackage{xspace}  % For spacing after user-defined commands and others

% This has to be loaded last because it modifies a lot of other commands.
\usepackage{hyperref}  % Allow hyperlinked references

% This option may fix a bug that sometimes causes acro to introduce unnecessary
% `s` to the start of acronyms. (https://bitbucket.org/cgnieder/acro/issues/120)
%\usepackage[display-foreign=false]{acro}

%\usepackage{algorithmic}  % Format algorithm pseudocode descriptions
%\usepackage{amsmath}  % cases environment
%\usepackage{amsfonts}  % Load AMS fonts
%\usepackage{amssymb}  % useful symbols (e.g., \varnothing)
%\usepackage{booktabs}  % Special top and bottom rules for tables
%\usepackage{bmpsize}  % Get bitmap size and resolutions

% Smaller captions. Aggressive space hack that should only be used if absolutely
% necessary.
%\usepackage[font=footnotesize]{caption}

%\usepackage{color}  % Used to highlight comments
%\usepackage[tmargin=1in, bmargin=1in, lmargin=0.75in, rmargin=0.75in,
%columnsep=0.25in]{geometry} % Exert more control over the margins
% \usepackage{mathptmx}  % Use Times New Roman font

% Tweak interword spacing and kerning. Aggressive space hack that should only be
% used if absolutely necessary.
%\usepackage[final,babel,step=1,stretch=30,shrink=40]{microtype}

%\usepackage{msc}  % Message sequence charts
%\usepackage{paralist}  % For tighter lists
%\usepackage{subfig}  % For multiple subfigures
%\usepackage{tabularx}  % For space-filling table columns
%\usepackage{textcomp}  % Various text symbols
%\usepackage[hyphens]{url}  % To make URLs display correctly in citations. The
                           % hyphens option allows line breaks after hyphens.

